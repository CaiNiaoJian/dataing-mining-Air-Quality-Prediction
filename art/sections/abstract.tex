\section*{摘要}

本研究旨在通过机器学习方法对空气质量进行预测分析。研究采用多种机器学习模型,包括随机森林、决策树等算法,对空气质量指标进行建模和预测。通过对大量历史数据的分析,我们发现气象因素(如温度、湿度、风向等)与空气质量之间存在显著的相关性。研究结果表明,所提出的预测模型能够有效捕捉空气质量的变化趋势,预测准确率达到了较高水平。特别是随机森林模型表现最为优异,其预测准确率达到90\%,均方根误差(RMSE)为0.15。此外,通过特征重要性分析,我们发现温度(TEMP)和气压(PRES)是影响空气质量的最关键因素。这项研究的成果可为空气质量预警和环境保护决策提供重要的参考依据。

\textbf{关键词:} 空气质量预测、机器学习、随机森林、数据挖掘、环境监测 