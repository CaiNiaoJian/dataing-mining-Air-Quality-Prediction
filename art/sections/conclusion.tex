\section{结束语}

本研究通过系统的数据挖掘方法,成功构建了一个准确可靠的空气质量预测模型。在多个机器学习模型的对比中,随机森林模型表现最为出色,预测准确率达到90\%,这充分证明了机器学习方法在环境监测领域的应用价值。特别是在特征工程方面,我们发现温度、气压和风速是影响空气质量的关键因素,这为进一步优化预测模型提供了重要依据。

通过这次数据挖掘实践,我深入理解了数据挖掘与机器学习算法的原理和应用,掌握了从数据预处理到模型评估的完整工作流程。在实践过程中,我不仅学会了使用Python进行数据分析和模型构建,还掌握了LaTeX的学术写作技能。这些经验对我今后在数据科学领域的学习和研究都将产生积极影响。尽管研究中仍存在一些局限性,但这也为未来的研究指明了方向,期待能在后续工作中不断改进和完善预测模型。

感谢祝老师为我提供的学习机会,让我有机会接触到数据挖掘与机器学习,并在此过程中收获了许多宝贵的经验。

